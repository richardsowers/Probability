\documentclass{beamer}

\usetheme{Luebeck}%boxes, Luebeck, Darmstadt

\usepackage{amssymb}%
\usepackage[mathscr]{eucal}
\usepackage{graphicx}
\usepackage{mathrsfs}
\usepackage{psfrag}
\usepackage{xypic}
\usepackage{url}
%\usepackage{xymatrix}
\setbeamercovered{dynamic}
\newcommand{\gen}{\mathscr{L}}
\newcommand{\filty}{\mathscr{Y}}

\usepackage[mathscr]{eucal}
\usepackage{times}
\usefonttheme{structurebold}
%
\usepackage[english]{babel}
\usepackage{pgf,pgfarrows,pgfnodes,pgfautomata,pgfheaps}
\usepackage{amsmath,amssymb}
\usepackage[latin1]{inputenc}
\usepackage{geometry}
%\geometry{landscape}


\DeclareGraphicsExtensions{.pdf, .jpg,.png,.jpeg}
\graphicspath{{images/}}
\theoremstyle{plain}

\newcommand{\R}{\mathbb{R}}
\newcommand{\N}{\mathbb{N}}
\newcommand{\lb}{\left\{}
\newcommand{\rb}{\right\}}
\newcommand{\Borel}{\mathscr{B}}
\newcommand{\Top}{\mathscr{T}}

\newcommand{\Err}{\mathscr{E}}

\newcommand{\eps}{\varepsilon}
\newcommand{\vtheta}{\vartheta}
\newcommand{\vkap}{\varkappa}

\newcommand{\BP}{\mathbb{P}}
\newcommand{\BE}{\mathbb{E}}
\newcommand{\KK}{K}

\newcommand{\Def}{\overset{\text{def}}{=}}


\newcommand{\Sone}{\mathfrak{S}^1}
\newcommand{\Cyl}{\mathfrak{C}}
\newcommand{\filt}{\mathscr{F}}
\newcommand{\gilt}{\mathscr{G}}
\newcommand{\wilt}{\mathscr{W}}
\newcommand{\la}{\left\langle}
\newcommand{\ra}{\right\rangle}


\newcommand{\bV}{\mathbf{V}}
\newcommand{\frakJ}{\mathfrak{J}}
\newcommand{\EGY}{\Delta}
\newcommand{\NN}{\mathcal{N}}
\newcommand{\bOne}{\mathbf{1}}
\newcommand{\der}{d}
\newcommand{\Gaussian}{\mathcal{G}}
\newcommand{\sa}{\mathscr{S}}
\newcommand{\TimeSet}{\mathcal{T}}



\title[]{Heavy Tails}
\author{Richard Sowers}
\institute[Illinois]{
  University of Illinois at Urbana-Champaign}
%\date{Fall 2018}


\begin{document}
\begin{frame}
\titlepage
\begin{center}
\includegraphics[width = 0.5\textwidth]{logo}
\end{center}
\end{frame}

\begin{frame}
We can \emph{invert} cumulatives;
\begin{equation*} F^{-1}(\alpha) \Def \inf\lb x\in \R:\ F(x)>\alpha\rb \end{equation*}

Let's now look at a \emph{QQ Plot}
\begin{center}\url{https://en.wikipedia.org/wiki/Q-Q_plot}\end{center}
Assume that we have two cumulatives $F_1$ and $F_2$, which are integrals of two positive density functions;
\begin{equation*} F_i(t)\Def \int_{s=-\infty}^t f_i(s)ds. \qquad i\in \{1,2\}, t\in \R \end{equation*}
Let's plot $(F_1^{-1}(\alpha),F_2^{-1}(\alpha))$ for $\alpha\in (0,1)$; 
this is a \emph{parametric plot}.
\end{frame}

\begin{frame}
Suppose that this plot is a line;
\begin{equation} \label{E:QQLine} F_2^{-1}(\alpha) = m F_1^{-1}(\alpha)+b \qquad \alpha\in (0,1) \end{equation}
Recalling how to take derivatives of inverse functions, we then have that
\begin{equation*} \frac{1}{f_2(F_2^{-1}(\alpha))} = \frac{m}{f_1(F_1^{-1}(\alpha))} \qquad \alpha\in (0,1) \end{equation*}
or rather, using \eqref{E:QQLine},
\begin{equation*} \frac{1}{f_2(m F_1^{-1}(\alpha)+b)} = \frac{m}{f_1(F_1^{-1}(\alpha))}\end{equation*}
which implies (you might take $x'=F_1^{-1}(\alpha)$) that 
\begin{equation*} f_1(x')=mf_2(mx'+b). \qquad x'\in \R\end{equation*}
If $X_1$ and $X_2$ are random variables with cumulatives, respectively, $F_1$ and $F_2$, this in turn implies that, in law, 
\begin{equation*}X_2=mX_1+b.\end{equation*}
\end{frame}

\begin{frame}
One often plots a reference line through the points corresponding to $\alpha=\tfrac14$ and $\alpha=\tfrac34$.
\end{frame}

\begin{frame}
Let's see what could happen with a \emph{heavy-tailed} distribution.

Assume that at the center of the QQ plot, $F_2^{-2}(\alpha)\approx F_1^{-1}(\alpha)$, so that near the center of the distribution $F_1$ and $F_2$ agree with each other.  Let's assume, however, that $F_2^{-1}(\alpha)\ll F_1^{-1}(\alpha)$ when $\alpha \approx 0$.  Setting
\begin{equation*} x_i\Def F_i^{-1}(\alpha) \end{equation*}
for $i\in \{1,2\}$, the inequality $x_2\ll x_1$ means that $F_2$ accumulates mass $\alpha$ further to the left than $F_1$.  The left tail of $F_2$ is thus \emph{fatter} than the left tail of $F_1$.

Similarly, let's assume that $F_2^{-1}(\alpha)\gg F_1^{-1}(\alpha)$ for $\alpha \approx 1$.
Again using \eqref{E:simple}, the inequality $x_2\gg x_1$ means that $F_2$ accumulates mass $\alpha$ further to the right than $F_1$, meaning that $F_2$ accumulates mass in the right tail at a more extreme value than $F_1$; the right tail of $F_2$ is thus \emph{fatter} than the right- tail of $F_1$.
\end{frame}
\end{document}




